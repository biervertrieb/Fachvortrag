% Inbuilt themes in beamer
\documentclass{beamer}

% Theme choice:
\usetheme{Frankfurt}

% Title page details: 
\title{Einführung in Microservices}
\author{Sebastian Heuer}
\institute[MLU - Informatik]{
    Institut für Informatik\\
    Martin-Luther-Universität Halle-Wittenberg
}
\date{03.03.2022}
\logo{\large \LaTeX{}}


\begin{document}

% Title page frame
\begin{frame}
    \titlepage
\end{frame}

% Remove logo from the next slides
\logo{}

\section{Zielsetzung}
\begin{frame}{Zielsetzung}
    \begin{itemize}
        \item Einführung in das Thema Microservices
        \item Überblick über Herausforderungen und Technologien
        \item Nachvollziehen eines Definitionsversuchs
        \item Wertung
    \end{itemize}
\end{frame}
\begin{frame}{Zielsetzung}
    Folgende Punkte werden (fast) nicht behandelt:
    \begin{itemize}
        \item Vor- und Nachteile
        \item Service-Komposition
        \item Frameworks, Docker und Kubernetes
    \end{itemize}
\end{frame}

\section{Einleitung}
\begin{frame}{Geschichte}
    \begin{itemize}
        \item 1999 - "resource oriented computing" - Dexter research project, Hewlett Packard Labs - REST als Teil von ROC abstraction
        \item 2005 - "Software components are Micro-Web-Services" - Dr. Peter Rodgers, Web Services Edge conference
        \item 2009 - Netflix spaltet den Monolithen und geht in die Cloud
        \item 2011 - "Microservices" - erste Erwähnung, workshop for software architects
        \item 2012 - "Microservices" - Namensgebung, workshop for software architects
        \item 2014 - "Microservices, a definition of this new architectural term" - Lewis Fowler, Fachartikel für Thoughtworks
        \item 2015 - "Building Microservices" - Buch von Sam Neil, Nachschlagewerk für design, development, deployment, testing und maintenance
    \end{itemize}
\end{frame}
\begin{frame}{Grundlagen}
    \begin{block}{Service-Oriented Architecture}
        Softwarekomponenten sind eigene Applikationen in einem verteilten System.
    \end{block}
    \begin{block}{Agile Entwicklung}
        \begin{itemize}
            \item Individuals and interactions over processes and tools
            \item Working software over comprehensive documentation
            \item Customer collaboration over contract negotiation
            \item Responding to change over following a plan
        \end{itemize}
    \end{block}
\end{frame}
\begin{frame}{Grundlagen}
    \begin{block}{DevOps}
        \begin{itemize}
            \item Methodensammlung zum Kombinieren von Entwicklung (Dev) und Organisation(Ops)
            \item Effektive Prozesse und Werkzeuge zur Steigerung der Qualität und Geschwindigkeit
            \item Betrifft: Entwicklung, Qualitätsmanagement, Auslieferung, Wartung, Kommunikation, ...
            \item Beispiele: Virtualisierung, Versionsverwaltung, Infrastructure as Code, Continuous Integration
        \end{itemize}
    \end{block}
    \begin{block}{CI/CD}
        Continuous Integration, Continuous Deployment\\
        Durchgehende Auslieferung von kleinen Änderungen anstatt Sammeln von Änderungen in größeren Updates
    \end{block}
\end{frame}
\begin{frame}{Motivation}
    Wir haben eine Sammlung von Anforderungen an die Softwareentwicklung
    \begin{block}{Organisation}
        \begin{itemize}
            \item Wir betrachten große, komplexe Anwendungen.
            \item Wir wollen kleine, autonome Teams.
        \end{itemize}

    \end{block}
    \begin{block}{Prozesse}
        \begin{itemize}
            \item Agile Methoden
            \item CI/CD
            \item DevOps
            \item ...
        \end{itemize}
    \end{block}
    \begin{block}{Architektur}
        \begin{itemize}
            \item Welcher Stil passt zu unseren Anforderungen?
        \end{itemize}
    \end{block}

\end{frame}

\section{Microservices}
\begin{frame}{Microservices}
\end{frame}
\begin{frame}{Microservices}
\end{frame}
\begin{frame}{Microservices}
\end{frame}
\begin{frame}{Kritik}
\end{frame}
\begin{frame}{Bewertung}
\end{frame}

\section{Definitionsversuch}
\begin{frame}{Versuch 1: Komponententrennung durch Services}
\end{frame}
\begin{frame}{Versuch 2: Zerlegen nach technischen Schichten}
\end{frame}
\begin{frame}{Versuch 3: Strukturierung nach Geschäftseinheiten}
\end{frame}
\begin{frame}{Bewertung der Definition}
\end{frame}

\section{Zusammenfassung}
\begin{frame}{Zusammenfassung}
    Microservices
    \begin{itemize}
        \item Beschreibung eines Architekturstils
        \item Keine formale Definition
        \item Definitionsversuch über gemeinsame Merkmale
        \item Aufbrechen von vorhandener Software entlang von geschäftsrelevanten Grenzen
    \end{itemize}
\end{frame}
\begin{frame}{Zusammenfassung}
    Microservices
    \begin{itemize}
        \item Geringe Kosten für Erweiterung, Skalierung
        \item Automatisierung von Deployment
        \item Hoher Einarbeitungsaufwand
    \end{itemize}
\end{frame}
\begin{frame}{Danke für Ihre Aufmerksamkeit}
    \centering Fragen?
    \\~\\
    \centering \includegraphics[scale=0.18]{cloud2.jpg}
\end{frame}

\appendix

\section{Zusatzmaterial}
\begin{frame}{Wann sollte man keine Microservices verwenden?}
    \begin{itemize}
        \item Geschäftsprozesse müssen bekannt sein. \\
              Änderungen in Service-Service Beziehungen sind teuer!
        \item Bei Neuentwicklungen sind die Abhängigkeiten zwischen Geschäftsprozessen nicht ausreichend bekannt. \\
              Man sollte mit einem Monolithen starten!
        \item Probleme die durch Microservices verursacht werden verschlimmern sich mit der Skalierung. \\
              Implementierung von Microservices sollte graduell passieren, je nach Änderungsvermögen des Systems!
    \end{itemize}
\end{frame}
\begin{frame}{Besondere Services}
    \begin{itemize}
        \item Ingress
        \item API Gateway
        \item Eventbus
        \item Dynamic Service Registry
        \item Management/Orchestration
    \end{itemize}
\end{frame}
\begin{frame}{Aufbau eines Microservices}
    \begin{itemize}
        \item Cluster IP
        \item Cache
        \item Security
        \item Safety
    \end{itemize}
\end{frame}

\end{document}